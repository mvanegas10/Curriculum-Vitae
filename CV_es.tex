\documentclass[10pt]{article}
\usepackage[margin=1in]{geometry}
\usepackage{titlesec}
\usepackage{multicol}
\usepackage{multirow}
\usepackage{hyperref}
\usepackage{multicol}
\usepackage[spanish]{babel}
\selectlanguage{spanish}
\usepackage[utf8]{inputenc}
\usepackage{amsmath, amssymb}
\usepackage{enumerate}
\usepackage{color}
\usepackage{tikz}
\titlespacing*{\section}
{0pt}{1em}{0.3em}

\pagenumbering{gobble}
\begin{document}



\begin{center}
\LARGE{Meili Vanegas Hernández} 

\normalsize{
 (+57) 318 489 2536 $\cdot$ Bogotá, Colombia
 
 \href{mailto:m.vanegas10@uniandes.edu.co}{\underline{m.vanegas10@uniandes.edu.co}}
 
\textcolor{blue}{\href{https://mvanegas10.github.io}{\underline{mvanegas10.github.io}}}
}
\end{center}
\hrule
\vspace{0.25cm}

\setcounter{secnumdepth}{0}
\setlength\parindent{0pt} 


{\large \textbf{Educación}}
{\footnotesize
\begin{multicols}{3}
{\bf Universidad de los Andes.} \\  \\ Bogotá, Colombia
\end{multicols}
 
\begin{itemize}
	 \item {\bf Maestría}  \qquad {\bf Ingeniería de Sistemas y Computación}  \qquad {\bf Enero 2017 - Diciembre 2018 (Esperado).} \qquad {\bf GPA: 4.33 (Escala 5.0)}
	 
	 \item {\bf Pregrado}  \qquad {\bf Ingeniería de Sistemas y Computación}  \qquad {\bf Julio 2012 - Diciembre 2016.} \qquad {\bf GPA: 4.08 (Escala 5.0)}
	
	\item \textbf{Cursos relevantes:} Visual Analytics, Imágenes y Visión, Inteligencia de Negocios, Agentes Inteligentes, Desarrollo de Aplicaciones Móviles, Infraestructura de Comunicaciones, Probabilidad y Estadística, Sistemas de Apoyo a la Decisión y Arquitectura Empresarial.  
		
	\item \textbf{Intereses:} Visual Analytics, Big Data, Inteligencia de Negocios, Desarrollo Web, Arte.
\end{itemize}}

\begin{multicols}{4}
{\bf Gimnasio Vermont.} \\  \\ Bogotá, Colombia
\end{multicols}
 
\begin{itemize}
	 \item {\bf Bachiller Summa Cum Laude}  \qquad {\bf Junio 2012}
\end{itemize}


\vspace{0.25cm}
\hrule
\vspace{0.25cm}

{\large \textbf{Experiencia profesional}}
{\footnotesize
\begin{multicols}{2}
{\bf Pasantía de investigación} \\ {\bf Université de Nice Sophia Antipolis. Niza, France}
\end{multicols}

\begin{itemize}
	\item \textcolor{blue}{\href{http://www.i3s.unice.fr/fr/presentation}{\underline{\textbf{Laboratorio I3S}}}} y \textcolor{blue}{\href{http://www.umrespace.org/?lang=fr}{\underline{\textbf{Laboratorio ESPACE}}}} (Junio 2016 - Julio 2016): Trabajé en el proyecto Transport Oriented Modeling for urban denSification Analysis (\textsc{TOMSA})/ECOS Nord, en el que se busca desarrollar herramientas para el apoyo a la toma de decisión en planeación urbana. Durante la estadía propuse un modelo urbano \textit{multiagente} con el que se puede simular la relocalización de hogares en una ciudad. Fue implementado usando Java, PostgreSQL, PostGIS, NodeJS, JavaScript, HTML y CSS.

\end{itemize}
\begin{multicols}{2}
{\bf Asistente de investigación} \\ {\bf Universidad de los Andes. Bogotá, Colombia}
\end{multicols}

\begin{itemize}
	\item \textcolor{blue}{\href{http://imagine.uniandes.edu.co}{\underline{\textbf{Grupo de Investigación IMAGINE}}}} (Enero 2016 - Diciembre 2016): Trabajé como investigadora junior en proyectos de analítica visual en sistemas urbanos. Desarrollé una aplicación web usando PostgreSQL, PostGIS, NodeJS, JavaScript, HTML y CSS.
\end{itemize}

 \begin{multicols}{2}
{\bf Asistente académico} \\ {\bf Universidad de los Andes. Bogotá, Colombia}
\end{multicols}

\begin{itemize}
    \item \textcolor{blue}{\href{https://sistemasacademico.uniandes.edu.co/~isis3301/dokuwiki/doku.php}{\underline{\textbf{Inteligencia de Negocios}}}} (Agosto 2016 - Diciembre 2016): Apoyé a la profesora María del Pilar Villamil en el curso, proponiendo diferentes escenarios para que los estudiantes aprendieran procesos de \textsc{ETL} básico, modelado dimensional y principios de \textit{Machine Learning}.
	\item \textcolor{blue}{\href{https://sistemasacademico.uniandes.edu.co/~isis1304/dokuwiki/doku.php}{\underline{\textbf{Fundamentos de Infraestructura Tecnológica}}}} (Enero 2015 - Diciembre 2015):
	Apoyé al profesor Rafael Gómez en el curso, enseñando bases de \textsc{C} y \textit{x86} \textit{Assembly} para que los estudiantes pudieran realizar actividades como cifrado de información en imágenes y sonido.
	\item \textcolor{blue}{\href{https://sistemasacademico.uniandes.edu.co/~isis1205/dokuwiki/doku.php}{\underline{\textbf{Algorítmica y Programación por Objetos II}}}} (Julio 2014 - Diciembre 2014):
	Apoyé al profesor John Casallas en el curso en el que se explican patrones de programación orientada a objetos, algoritmos de búsqueda y ordenamiento, uso de árboles binarios, entre otros.
	\item \textcolor{blue}{\href{http://catalogo.uniandes.edu.co/2014/Catalog/Courses/IIND/2000/IIND-2109}{\underline{\textbf{Sistemas de Apoyo a la Decisión}}}} (Julio 2013 - Mayo 2015):
	Apoyé a diferentes profesores en la materia, en la que se enseñan herramientas básicas de Excel, Visual Basic y Access.
\end{itemize}

\vspace{0.25cm}
\hrule
\vspace{0.25cm}
{\large \textbf{Experiencia técnica}}
\vspace{0.25cm}
{\footnotesize\begin{itemize}

\item \textcolor{blue}{\href{http://bl.ocks.org/mvanegas10/raw/3bd6d116da4ed96d212e783060720c5d/}{\underline{\textbf{Así es el país que votó No:}}}} [JavaScript, Python, Jupyter Notebooks, \textsc{HTML}, \textsc{CSS}]  Aplicación web que permite visualizar la correlación de variables de resultados electorales y características de los municipios con los resultados del plebiscito del 2016 en Colombia. Código fuente: \textcolor{blue}{\href{https://github.com/mvanegas10/Plebiscito-Colombia-2016}{\underline{Github}}}.

\item \textcolor{blue}{\href{https://github.com/mvanegas10/ErosionIdentificationFromLandsatImages}{\underline{\textbf{Identificación de erosión desde imágenes Landsat:}}}} [Python] Proyecto de procesamiento y análisis de imágenes que permite identificar erosión en el departamento del Cesar, Colombia por medio de imágenes \textit{Landsat} 8. Código fuente: \textcolor{blue}{\href{https://github.com/mvanegas10/ErosionIdentificationFromLandsatImages}{\underline{Github}}}.

\item \textcolor{blue}{\href{https://github.com/mvanegas10/kobdig-validation}{\underline{\textbf{Modelo urbano basado en agentes:}}}} [NodeJS, Java, PostgreSQL, PostGIS]  Modelo urbano que permite simular la relocalización de hogares en una ciudad bajo un contexto espacial, económico y posibilístico. Código fuente: \textcolor{blue}{\href{https://github.com/mvanegas10/kobdig-validation}{\underline{Github}}}.

\item \textcolor{blue}{\href{https://github.com/mvanegas10/DataEncryption}{\underline{\textbf{Cifrado de información en imágenes y sonido:}}}} [C \& x86 Assembly] Sistema de cifrado de información manipulando bits, usando punteros y administrando la memoria dinámicamente para guardar la infomación en imágenes y sonido. Código fuente: \textcolor{blue}{\href{https://github.com/mvanegas10/DataEncryption}{\underline{Github}}}.
\end{itemize}}

\vspace{0.25cm}
\hrule
\vspace{0.25cm}
{\large \textbf{Idiomas y tecnologías}}
\vspace{0.25cm}
{\footnotesize
\begin{itemize}
	\item Español (Lengua materna), Inglés (\textsc{TOEFL iBT} 90/120), Francés (Básico A1)
	\item Python, JavaScript, C, SQL, Swift, \textsc{Matlab}, Java, Visual Basic, \textsc{HTML}, \textsc{CSS}.
	\item NodeJS, D3.js, C3.js, Leaflet, QGIS, Sublime, \LaTeX, PostgreSQL, PostGIS, Jupyter Notebooks, Django, IntelliJ IDEA, Play Framework, Tableau, Visual Studio, Eclipse, Netbeans, Adobe Illustrator.
\end{itemize}}

\vspace{0.25cm}
\hrule
\vspace{0.25cm}
{\large \textbf{Reconocimientos}}
\vspace{0.25cm}
{\footnotesize
\begin{itemize}
    \item \textcolor{blue}{\href{http://alianzacaoba.co/alianza-caoba/hackathon2016/} {\underline{ Hackathon IBM, Dirección Nacional de Planeación de Colombia \textsc{(DNP)}, Universidad de los Andes y Alianza CAOBA}}} \textbf{(Premiada), 2016:} Mi equipo fue premiado en la Hackathon Cognitiva por la propuesta analítica sobre datos abiertos del gobierno colombiano.
	\item \textbf{Concurso de innovación en TI (Finalista), Universidad de los Andes, 2015:} Mi grupo quedó en segundo lugar en el \textit{Concurso de Innovación en Tecnologías de la Información} de la Universidad de los Andes por el sistema de predicción en juegos de fútbol \textit{betgram}.
	\item \textbf{Summa Cum Laude, Gimnasio Vermont, 2012:} Recibí el grado de bachiller con honores por mi monografía en matemáticas llamada \textit{Búsqueda de un algoritmo para calcular la raíz enésima de cualquier número real}.
	\item \textbf{International Baccalaureate, 2012:} Finalicé el Programa del Diploma (\textit{International Baccalaureate Diploma Programme}) con una valoración total de 27/45.
\end{itemize}}

\end{document}