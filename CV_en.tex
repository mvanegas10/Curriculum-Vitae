\documentclass[10pt]{article}
\usepackage[margin=1in]{geometry}
\usepackage{titlesec}
\usepackage{multicol}
\usepackage{multirow}
\usepackage{hyperref}
\usepackage{multicol}
\usepackage[spanish]{babel}
\selectlanguage{spanish}
\usepackage[utf8]{inputenc}
\usepackage{amsmath, amssymb}
\usepackage{enumerate}
\usepackage{color}
\usepackage{tikz}
\titlespacing*{\section}
{0pt}{1em}{0.3em}

\pagenumbering{gobble}
\begin{document}



\begin{center}
\LARGE{Meili Vanegas Hernández} 

\normalsize{
 (+57) 318 489 2536 $\cdot$ Bogotá, Colombia
 
 \href{mailto:m.vanegas10@uniandes.edu.co}{\underline{m.vanegas10@uniandes.edu.co}}
 
\textcolor{blue}{\href{https://mvanegas10.github.io}{\underline{mvanegas10.github.io}}}

}
\end{center}
\hrule
\vspace{0.25cm}

\setcounter{secnumdepth}{0}
\setlength\parindent{0pt} 


{\large \textbf{Education}}
{\footnotesize
\begin{multicols}{3}
{\bf Universidad de los Andes.} \\  \\ Bogotá, Colombia
\end{multicols}
 
\begin{itemize}
	 \item {\bf Master's Degree}  \qquad {\bf Systems \& Computing Engineering}  \qquad {\bf January 2017 - December 2018 (Expected).} \qquad {\bf GPA: 4.33 (5.0 scale)}
	 
	 \item {\bf Bachelor's Degree}  \qquad {\bf Systems \& Computing Engineering}  \qquad {\bf July 2012 - December 2016.} \qquad {\bf GPA: 4.08 (5.0 scale)}	
		
	\item \textbf{Interests:} Visual Analytics, Big Data, Computer Networking, Image Analysis and Processing, Business Intelligence, Urban Systems, Arts.
\end{itemize}}

\begin{multicols}{4}
{\bf Gimnasio Vermont.} \\  \\ Bogotá, Colombia
\end{multicols}
 
\begin{itemize}
	 \item {\bf High School, Summa Cum Laude}  \qquad {\bf June 2012}
\end{itemize}


\vspace{0.25cm}
\hrule
\vspace{0.25cm}

{\large \textbf{Professional Experience}}
{\footnotesize
\begin{multicols}{2}
{\bf Research Internship} \\ {\bf Université de Nice Sophia Antipolis. Nice, France}
\end{multicols}

\begin{itemize}
	\item \textcolor{blue}{\href{http://www.i3s.unice.fr/fr/presentation}{\underline{\textbf{I3S Laboratory}}}} and \textcolor{blue}{\href{http://www.umrespace.org/?lang=fr}{\underline{\textbf{ESPACE Laboratory}}}} (June 2016 - July 2016): Working in the project Transport Oriented Modeling for urban denSification Analysis (TOMSA)/ECOS Nord. Building a urban decision support platform, which holds a simulation based on an multi-agent urban model of densification implemented in Java and a visual analytics tool using PostgreSQL, PostGIS, NodeJS, JavaScript, HTML and CSS.

\end{itemize}
\begin{multicols}{2}
{\bf Research Assistant} \\ {\bf Universidad de los Andes. Bogotá, Colombia}
\end{multicols}

\begin{itemize}
	\item \textcolor{blue}{\href{http://imagine.uniandes.edu.co}{\underline{\textbf{IMAGINE research team}}}} (January 2016 - December 2016): Working as undergraduate researcher for the image processing research team at Universidad de los Andes, performing visual analytics in urban systems, centered on analyzers and researchers as end users. Developing a web application using PostgreSQL, PostGIS, NodeJS, JavaScript, HTML and CSS.
\end{itemize}

 \begin{multicols}{2}
{\bf Teaching Assistant} \\ {\bf Universidad de los Andes. Bogota, Colombia}
\end{multicols}

\begin{itemize}
    \item \textcolor{blue}{\href{https://sistemasacademico.uniandes.edu.co/~isis3301/dokuwiki/doku.php}{\underline{\textbf{Business Intelligence}}}} (August 2016 - December 2016):
	Teaching basic analytics. Proposed different scenarios to help students learn basics of \textsc{ETL} processes, dimensional modelling and Machine Learning.
	\item \textcolor{blue}{\href{https://sistemasacademico.uniandes.edu.co/~isis1304/dokuwiki/doku.php}{\underline{\textbf{Computer organization}}}} (January 2015 - December 2015):
	Teaching basic computer organization, its components, functions and interaction. Designed project assignments to help students learn basic x86 Assembly language programming and  C, to perform tasks such as data encryption in sound and image media. 
	\item \textcolor{blue}{\href{https://sistemasacademico.uniandes.edu.co/~isis1205/dokuwiki/doku.php}{\underline{\textbf{Object Oriented Programming II}}}} (July 2014 - December 2014):
	Teaching Object-Oriented Programming patterns, implementing searching and sorting algorithms in Java, recursions, and the use of binary trees and other basic data structures to solve practical CS problems, and Java GUI programming.
	\item \textcolor{blue}{\href{http://catalogo.uniandes.edu.co/2014/Catalog/Courses/IIND/2000/IIND-2109}{\underline{\textbf{Decision support systems}}}} (July 2013 - May 2015):
	Teaching decision support using computing tools. Modeling and simulating real industrial engineering problems in Visual Basic.
\end{itemize}



\vspace{0.25cm}
\hrule
\vspace{0.25cm}
{\large \textbf{Technical experience}}
\vspace{0.25cm}
{\footnotesize\begin{itemize}

\item \textcolor{blue}{\href{http://bl.ocks.org/mvanegas10/raw/3bd6d116da4ed96d212e783060720c5d/}{\underline{\textbf{Así es el país que votó No:}}}} [JavaScript, Python, Jupyter Notebooks, \textsc{HTML}, \textsc{CSS}]  Visual Analytics tool that shows the correlation of demographic variables over different towns of Colombia and the results of the National Peace Agreement Referendum of 2016. Public available code: \textcolor{blue}{\href{https://github.com/mvanegas10/Plebiscito-Colombia-2016}{\underline{Github}}}.

\item \textcolor{blue}{\href{https://github.com/mvanegas10/ErosionIdentificationFromLandsatImages}{\underline{\textbf{Erosion identification from Landsat images:}}}} [Python] Image processing using satellite acquired images to identify erosion in mining regions in Colombia. Public available code: \textcolor{blue}{\href{https://github.com/mvanegas10/ErosionIdentificationFromLandsatImages}{\underline{Github}}}.

\item \textcolor{blue}{\href{https://github.com/mvanegas10/kobdig-validation}{\underline{\textbf{Urban Agent-Based Model:}}}} [NodeJS, Java, PostgreSQL, PostGIS] Proposed a Urban Agent-Based Model \textsc(ABM) to simulate the relocation of households under a spatial and possibilistic scenario. Public available code: \textcolor{blue}{\href{https://github.com/mvanegas10/kobdig-validation}{\underline{Github}}}.

\item \textcolor{blue}{\href{https://github.com/mvanegas10/DataEncryption}{\underline{\textbf{Data encryption over multimedia:}}}} [C \& x86 Assembly] Data encryption system development, using bit manipulation, pointers, and dynamic memory management to manipulate data and save it on images and sound. Public available code: \textcolor{blue}{\href{https://github.com/mvanegas10/DataEncryption}{\underline{Github}}}.
\end{itemize}}

\vspace{0.25cm}
\hrule
\vspace{0.25cm}
{\large \textbf{Languages and technologies}}
\vspace{0.25cm}
{\footnotesize
\begin{itemize}
	\item Spanish (Native), English (\textsc{TOEFL iBT} 90/120), French (Basic A1)
	\item Python, JavaScript, C, SQL, Swift, \textsc{Matlab}, Java, Visual Basic, \textsc{HTML}, \textsc{CSS}.
	\item NodeJS, D3.js, C3.js, Leaflet, QGIS, Sublime, \LaTeX, PostgreSQL, PostGIS, Jupyter Notebooks, Django, IntelliJ IDEA, Play Framework, Tableau, Visual Studio, Eclipse, Netbeans, Adobe Illustrator.
\end{itemize}}

\vspace{0.25cm}
\hrule
\vspace{0.25cm}
{\large \textbf{Achievements}}
\vspace{0.25cm}
{\footnotesize
\begin{itemize}
    \item \textcolor{blue}{\href{http://alianzacaoba.co/alianza-caoba/hackathon2016/} {\underline{ Hackathon IBM, Dirección Nacional de Planeación de Colombia \textsc{(DNP)}, Universidad de los Andes and Alianza CAOBA}}} \textbf{(Winner), 2016:} My team won the IBM's Hackathon Cognitiva in which we proposed a Visual Analytics tool with Colombian's government open data. 
	\item \textbf{IT Innovation Contest (Finalist), Universidad de los Andes, 2015:} Won 2nd place in \textit{Concurso de Innovación en TI (IT Innovation Contest)} at Universidad de los Andes for our soccer score prediction system \textit{betgram}.
	\item \textbf{Summa Cum Laude, Gimnasio Vermont, 2012:} Received this award for the development of my extended essay in mathematics. The research aimed to pursuit an algorithm to calculate the \textit{nth} root of any real number.
	\item \textbf{International Baccalaureate, 2012:} Finalized the two-year \textit{International Baccalaureate Diploma Programme} with a total score of 27/45.
\end{itemize}}

\end{document}